% Chapter 1

\chapter{Introduction} % Main chapter title

\label{introduction} % For referencing the chapter elsewhere, use \ref{Chapter1} 

\lhead{Chapter 1. \emph{Introduction}} % This is for the header on each page - perhaps a shortened title

\par 
\textit{Sentiment Analysis} is the technique of detecting sentiment/opinion behind a \textit{word, sentence, collection of sentences, documents,} 
and even a \textit{collection of documents} in some cases. Here, \textit{word, sentence, collection of sentences, documents,} and \textit{collection of 
documents} can be termed as chunks of text which determines granularity of the analysis. We can make use of the general term 
\emph{text} when the discussion applies to all these chunks and specify the exact granularity when required. \textit{Sentiment Analysis} 
might also involve classifying \emph{text} as either \emph{Objective} (factual information) or \emph{Subjective} (expressing some 
sentiment or opinion). This is called as \textit{Subjectivity Analysis}. It can also be considered as preprocessing for \textit{Sentiment Analysis} in some cases. 
But \textit{Subjectivity Analysis} is considered a task within \textit{Sentiment Analysis}. \textit{Sentiment analysis} is also known as \textit{Opinion 
Mining} and these two terms are used quite interchangeably.

\par

In most cases, \textit{Sentiment Analysis} is a binary classification task in which a \emph{text} is classified as either positive or negative.
Examples of binary classification are \textit{movie reviews, product reviews, etc}. \textit{Ternary Classification}, wherein the \emph{text} is classified
as positive, negative or objective also has many applications. 

\par

This field is considerably new and is gaining of lot of attention. \textit{Movie reviews} of critics are classified as positive or negative
by using this technique. Same is the case with product reviews. \textit{tweets, comments, etc.} are analyzed to detect the positive or negative
sentiment behind them and sentiment about a particular entity. \textit{IR} also makes use of \textit{SA} these days to filter out subjective information
and retrieve only the objective data. There is also a motivation for sentiment aware IR in which documents of relevant sentiment 
(either positive or negative) are fetched. 

\section{Motivation}
\par
The \textit{Motivation} behind this research is to study sentiment analysis in general and the how can some of the techniques used in information 
retrieval be applied to it in particular. The unrealistic assumption made before applying this technique in many cases is that subjective data is available. 
Subjective text has to be retrieved from the web. Also, most of the data available is domain dependent. This makes most of the supervised methods not 
feasible for domain independent sentiment analysis. Thus, there is a need for a semi-supervised or unsupervised method which is domain independent. The 
fact that most of the previous works take into account all the sentiment-bearing words and phrases is the motivation behind using deep semantics.


\section{Problem Statement}
\par
The aim of this project is to use different approaches like topic modeling and deep semantics for sentiment analysis. To achieve the first goal different topic
models need to be used for \textit{SA} and compared with each other on some standard data. To achieve the second goal, some sort of deep semantic processing 
has to be performed. This semantic information is then to be used to perform sentiment classification.

\section{Contributions}
\par
Following contributions were made during the course of this project.
\begin{enumerate}
 \item Studied \textit{LDA}, \textit{JST}, and \textit{Topical n-grams} models.
 \item Evaluated \textit{LDA} for document clustering.
 \item Used LDA and Topical n-gram model for binary sentiment classification.
 \item Evaluated \textit{LDA, JST,} and \textit{Topical n-gram model} for binary sentiment classification task.
 \item Proposed an approach for generation of positive and negative word lists using \textit{LDA}.
 \item Developed a rule-based system for ternary sentiment classification using \textit{UNL}.
\end{enumerate}

%----------------------------------------------------------------------------------------

\section{Roadmap}\label{sec:roadmap}
% To be changed later
\par
This report gives the problem statement and the contributions in \cref{introduction}. \cref{sa} starts with a psychological viewpoint of \textit{sentiment}. 
This is followed by formal problem definition, and then types of \textit{SA}, challenges in \textit{SA}, and some applications of \textit{SA} are
listed. \cref{sa} also explains basic machine learning techniques and their applications in \textit{SA}. \cref{ir} starts with basics of \textit{IR}. 
Then it moves on to discuss the various \textit{text modeling} approaches prevalent in \textit{IR} with a focus on \textit{LDA}. It also shows how \textit{SA}
can be used for sentiment aware \textit{IR}. \cref{topicmodeledsa} discusses two models which combine sentiment and topics. The pros and cons of these models have 
been discussed. In \cref{topicalngram}, the topical n-grams models is explained and a technique to use it for the sentiment classification task is expalined. 
\cref{unl} focuses on the second aspect of this research which is make use of deep semantics for sentiment analysis. \cref{experiments} describes the experiments 
performed using the techniques described in \cref{topicmodeledsa}, \cref{topicalngram}, and \cref{unl}. \cref{conclusions} concludes and hints on some future work.

\section*{SUMMARY}

In this chapter, the problem statement was introduced along with the motivation. The significant contributions done so far have been listed.
The structure of the report was described in the \sref{sec:roadmap}.

In the next chapter we will learn \textit{Sentiment Analysis} and several machine learning techniques and how these are used in \textit{Sentiment Analysis}.

\clearpage


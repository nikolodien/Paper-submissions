% Chapter 1

\chapter{Introduction} % Main chapter title

\label{Chapter1} % For referencing the chapter elsewhere, use \ref{Chapter1} 

\lhead{Chapter 1. \emph{Introduction}} % This is for the header on each page - perhaps a shortened title

\par 
\textit{Sentiment Analysis} is the technique of detecting sentiment/opinion behind a \textit{word, sentence, collection of sentences, documents,} 
and even a \textit{collection of documents} in some cases. Here, \textit{word, sentence, collection of sentences, documents,} and \textit{collection of 
documents} can be termed as chunks of text which determines granularity of the analysis. We can make use of the general term 
\emph{text} when the discussion applies to all these chunks and specify the exact granularity when required. \textit{Sentiment Analysis} 
might also involve classifying \emph{text} as either \emph{Objective} (factual information) or \emph{Subjective} (expressing some 
sentiment or opinion). This is called as \textit{Subjectivity Analysis}. It can also be considered as preprocessing for \textit{Sentiment Analysis} in some cases. 
But \textit{Subjectivity Analysis} is considered a task within \textit{Sentiment Analysis}. \textit{Sentiment analysis} is also known as \textit{Opinion 
Mining} and these two terms are used quite interchangeably.

\par

In most cases, \textit{Sentiment Analysis} is a binary classification task in which a \emph{text} is classified as either positive or negative.
Examples of binary classification are \textit{movie reviews, product reviews, etc}. \textit{Ternary Classification}, wherein the \emph{text} is classified
as positive, negative or objective also has many applications. 

\par

This field is considerably new and is gaining of lot of attention. \textit{Movie reviews} of critics are classified as positive or negative
by using this technique. Same is the case with product reviews. \textit{tweets, comments, etc.} are analyzed to detect the positive or negative
sentiment behind them and sentiment about a particular entity. \textit{IR} also makes use of \textit{SA} these days to filter out subjective information
and retrieve only the objective data. There is also a motivation for sentiment aware IR in which documents of relevant sentiment 
(either positive or negative) are fetched. 

\section{Motivation}
\par
The \textit{Motivation} behind this research is to study sentiment analysis in general and the role of sentiment analysis in information retrieval 
in particular. \textit{Sentiment Analysis} has lot of applications as discussed. But, the assumption made before applying this technique in
many cases is that subjective data is available. This assumption is unrealistic. Subjective text has to be retrieved from the web. But, most
of the algorithms in \textit{Information Retrieval} are designed to fetch information relevant to a specific topic or a topic set. If 
\textit{Subjectivity Analysis} is combined with \textit{IR} then we can fetch subjective \textit{text}. Also, retrieval of \textit{text} of 
a specific sentiment is required in many applications. Thus, combining \textit{SA} with \textit{IR} to serve the needs of many applications
is the motivation behind this report.


\section{Problem Statement}
\par
The aim of this project is to aid retrieval of subjective \textit{text} of the desired sentiment by making use of \textit{Sentiment Analysis}
in \textit{Information Retrieval}. To achieve this a joint model of topic and sentiment has to be designed, implemented and compared with
other existing approaches. Also, the use of this model in sentiment classification of \textit{text} has to be evaluated.

\section{Contribution}
\par
Following contributions have been made till now:
\begin{enumerate}
 \item Implemented two systems using \citep*{apachelucene}, \citep*{sentiwordnet}, and \citep*{stanfordpostagger} for sentiment aware information
 retrieval to get a gist of the problem.
 \item Studied \textit{LDA}, Gibbs Sampling, and Inferencing using \textit{LDA}.
 \item Evaluated \textit{LDA} using implementation in \citep*{mallet}.
 \item Studied several joint sentiment topic models.
\end{enumerate}

%----------------------------------------------------------------------------------------

\section{Roadmap}\label{sec:roadmap}
% To be changed later
\par
This report gives the problem statement and the contributions in \cref{Chapter1}. \cref{Chapter2} starts with a psychological viewpoint of \textit{sentiment}. 
This is followed by formal problem definition, and then types of \textit{SA}, challenges in \textit{SA}, and some applications of \textit{SA} are
listed. \cref{Chapter2} also explains basic machine learning techniques and their applications in \textit{SA}. \cref{Chapter3} starts
with basics of \textit{IR}. Then it moves on to discuss the various \textit{text modeling} approaches prevalent in \textit{IR} with a focus
on \textit{LDA}. \cref{Chapter4} discusses two models which combine sentiment and topics. The pros and cons of these models have been discussed. 
\cref{Chapter5} describes two systems which make use of sentiment analysis in information retrieval. Also, an experiment to evaluate \textit{LDA}
is described. \cref{Chapter6} concludes and hints on some future work.

\section*{SUMMARY}

In this chapter, the problem statement was introduced along with the motivation. The significant contributions done so far have been listed.
The structure of the report has been described in the \sref{sec:roadmap}.

In the next chapter we will learn \textit{Sentiment Analysis} and several machine learning techniques and how these are used in \textit{Sentiment Analysis}.

\clearpage


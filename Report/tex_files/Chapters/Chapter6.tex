% Chapter 6

\chapter{Conclusions and Future Work} % Main chapter title

\label{Chapter6} % For referencing the chapter elsewhere, use \ref{Chapter6} 

\lhead{Chapter 6. \emph{Conclusions and Future Work}} % This is for the header on each page - perhaps a shortened title

%----------------------------------------------------------------------------------------

\section{Conclusion}

In this report we have learned that a wide variety of approaches have been used for detecting sentiment in \textit{text}. Supervised, Unsupervised 
and semi-supervised methods have been explored to solve this difficult problem. Due the unstructured nature of the \textit{text} used for
sentiment analysis, standard \textit{NLP} tools cannot be used directly. This has lead many techniques to make use of bag-of-words like models which 
makes strong independence assumptions. The accuracy of bag-of-words model is less than desirable. So, this model is augmented with information
about coherency, discourse, and pragmatics if it is present in the \textit{text}. Graph based formulation has become really important to encode 
this information in the feature vector as it is very difficult to encode it in the standard \textit{ML} techniques discussed. Domain dependence of sentiment
also plays a very crucial role in sentiment analysis. A complete reversal of sentiment orientation is observed in some cases. Problems like 
sarcasm and thwarting still persist and no clear solution is present at the moment. 

\textit{Information Retrieval} is mostly concerned with factual data. Using subjectivity analysis, subjective \textit{text} can be filtered out. This
can remove \textit{text} containing opinion of the author. But in some cases, retrieval of subjective text is very important. For example, you might want to fetch
all reviews related to a product. This is called sentiment aware \textit{IR}. In information retrieval, corpus modeling has been used extensively over the years to fetch
documents related to a specific topics. The topics of interest are expressed by the words in the query. Topic modeling approach can be extended
to model the sentiment of the \textit{text}. When these two models are combined, not only does it help in fetching \textit{text} pertaining 
to certain topics but also of the desired sentiment. Though these models make strong independence assumptions, their combination helps to 
model topic dependence of sentiment by encoding co-occurrence between sentiment and topic words. \textit{Sentiment Analysis} and \textit{Information Retrieval}
can be combined to achieve very good results in sentiment aware information retrieval.

Joint modeling of a sentiment and topics can also be used for sentiment classification. The sentiment which has highest proportion will be
the sentiment of the \textit{text}.

\section{Future work}

\subsection{Short term}
Designing a new generative model for sentiment and topics which is more intuitive that \textit{JST}. \textit{JST} doesn't exactly depict
the process of document generation. The fact that the topic chosen depends on the sentiment is not intuitive. So, designing a new model 
which addresses this problem is necessary.

\subsection{Medium term}
Implementing the model and using it for sentiment classification. Comparing it with existing approaches for sentiment classification.

\subsection{Long term}

Unstructured nature of the \textit{text} mostly involved in sentiment analysis is a very big obstacle. This makes it difficult
to take into account discourse and pragmatics. There is shortage of work which consider these important elements of textual data.
Graph based solutions to these problems have been discussed. Most of the future work in \textit{SA} will focus on how to integrate this
information in the feature vector so that it results in better classification.

Strong independence assumptions as we saw might mislead the classifier. Instead of using the same type of corpus modeling used 
in many \textit{IR} applications, effort should be made to embed discourse and pragmatics information into the model. This will improve
the accuracy of sentiment aware \textit{IR} systems. 

% Chapter 6

\chapter{Conclusions and Future Work} % Main chapter title

\label{conclusions} % For referencing the chapter elsewhere, use \ref{Chapter6} 

\lhead{Chapter 6. \emph{Conclusions and Future Work}} % This is for the header on each page - perhaps a shortened title

%----------------------------------------------------------------------------------------

\section{Conclusion}

% A wide variety of approaches have been used for detecting sentiment in \textit{text}. Supervised, Unsupervised and semi-supervised methods have been 
% explored to solve this difficult problem. Due the unstructured nature of the \textit{text} used for sentiment analysis, standard \textit{NLP} tools 
% cannot be used directly. This has lead many techniques to make use of bag-of-words like models which makes strong independence assumptions. The accuracy
% of bag-of-words model is less than desirable. So, this model is augmented with information about coherency, discourse, and pragmatics if it is present 
% in the \textit{text}. Graph based formulation has become really important to encode this information in the feature vector as it is very difficult to 
% encode it in the standard \textit{ML} techniques discussed. Domain dependence of sentiment also plays a very crucial role in sentiment analysis. A complete
% reversal of sentiment orientation is observed in some cases. Problems like sarcasm and thwarting still persist and no clear solution is present at the moment. 
% 
% \textit{Information Retrieval} is mostly concerned with factual data. Using subjectivity analysis, subjective \textit{text} can be filtered out. This can 
% remove \textit{text} containing opinion of the author. But in some cases, retrieval of subjective text is very important. For example, you might want to 
% fetchall reviews related to a product. This is called sentiment aware \textit{IR}. In information retrieval, corpus modeling has been used extensively over
% the years to fetchdocuments related to a specific topics. The topics of interest are expressed by the words in the query. Topic modeling approach can be 
% extended to model the sentiment of the \textit{text}. When these two models are combined, not only does it help in fetching \textit{text} pertaining to certain
% topics but also of the desired sentiment. Though these models make strong independence assumptions, their combination helps to model topic dependence of 
% sentiment by encoding co-occurrence between sentiment and topic words. \textit{Sentiment Analysis} and \textit{Information Retrieval} can be combined to 
% achieve very good results in sentiment aware information retrieval.

Topical n-gram model can be used for the binary sentiment classification problem. The motivation behind this approach was to use not only words but also 
phrases to classify documents. The model was trained with documents containing only subjective and negation words and bigrams formed by the combination of 
these. It also made use of rules to assign topics to words and bigrams in the intialization stage of estimation. Results using a prior show statistically 
significant improvement over the JST model. The paper also shows the use of LDA for resource generation, prompted by observations made during error analysis. 
The system shows marginal improvement in accuracy after using the resources generated by this technique.

Using deep semantics for sentiment analysis of structured data increases classification accuracy sentiment analysis. A semantic role labeling method through 
generation of a UNL graph was used to do this. The main motivation behind this research was the fact that not all sentiment bearing expressions contribute to
the overall sentiment of the \textit{text}. The approach was evaluated on two datasets and compared with successful previous approaches which don't make use 
of deep semantics. The system underperformed all the supervised systems but showed promise by yielding better results than the other rule-based approach. Also, 
in some cases the performance was very close to the other supervised systems. The system works well on sentences where are inherently complex and difficult 
for sentiment analysis as it makes use of semantic role labeling. Any rule based system can never be exhaustive in terms of rules. We always need to add new 
rules to improve on it. In some case, adding new rules might cause side-effects. In this case, as the rules are intuitive, adding of new rules will be easy. 
Also, analysis of the results hints at some ways to tackle specific problems effectively.

\section{Future work}

\subsection{Short term}
The system using topical n-grams model is focused on bigrams at the moment. It can be extended to handle n-grams by adding rules to detect and assign topics. 
Instead of adding rules, we can make use of machine learning techniques to detect the subjective nature of a phrase. For the \textit{UNL} system, Adding more 
rules to the system will help to improve the system. Language gets updated almost daily, we plan to update our dictionary with these new words and expressions 
to increase the accuracy. Also, we plan to replace the UNL system with a dependency parsing system and apply rules similar to the ones described in this work.

\subsection{Medium term}
Use of topic models and semantic information, to perform the aspect based sentiment analysis. We can identify the various terms using deep semantics and their 
associated sentiment can be found out using rules. The sentiment for each term can be used to find out the sentiment of each aspect by the use of topic models.

\subsection{Long term}
A combined model of topic models and semantics for binary sentiment classification.

% Unstructured nature of the \textit{text} mostly involved in sentiment analysis is a very big obstacle. This makes it difficult
% to take into account discourse and pragmatics. There is shortage of work which consider these important elements of textual data.
% Graph based solutions to these problems have been discussed. Most of the future work in \textit{SA} will focus on how to integrate this
% information in the feature vector so that it results in better classification.
% 
% Strong independence assumptions as we saw might mislead the classifier. Instead of using the same type of corpus modeling used 
% in many \textit{IR} applications, effort should be made to embed discourse and pragmatics information into the model. This will improve
% the accuracy of sentiment aware \textit{IR} systems. 

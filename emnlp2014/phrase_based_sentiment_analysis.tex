%
% File acl2014.tex
%
% Contact: koller@ling.uni-potsdam.de, yusuke@nii.ac.jp
%%
%% Based on the style files for ACL-2013, which were, in turn,
%% Based on the style files for ACL-2012, which were, in turn,
%% based on the style files for ACL-2011, which were, in turn, 
%% based on the style files for ACL-2010, which were, in turn, 
%% based on the style files for ACL-IJCNLP-2009, which were, in turn,
%% based on the style files for EACL-2009 and IJCNLP-2008...

%% Based on the style files for EACL 2006 by 
%%e.agirre@ehu.es or Sergi.Balari@uab.es
%% and that of ACL 08 by Joakim Nivre and Noah Smith

\documentclass[11pt]{article}
\usepackage{acl2014}
\usepackage{times}
\usepackage{url}
\usepackage{latexsym}

%\setlength\titlebox{5cm}

% You can expand the titlebox if you need extra space
% to show all the authors. Please do not make the titlebox
% smaller than 5cm (the original size); we will check this
% in the camera-ready version and ask you to change it back.


\title{Phrase based Sentiment Analysis using topical n-grams}

% \author{Nikhilkumar Jadhav \\
%   Masters Student \\
%   Computer Science \& Engg. Dept. \\
%   IIT Bombay \\
%   {\tt nikhilkumar@cse.iitb.ac.in} \\\And
%   Pushpak Bhattacharyya \\
%   Professor \\
%   Computer Science \& Engg. Dept. \\
%   IIT Bombay \\
%   {\tt pb@cse.iitb.ac.in} \\}

\date{}

\begin{document}
\maketitle
\begin{abstract}
Sentiment Analysis is mainly the classification of text into two classes viz. 
positive and negative. Joint sentiment and topic models have been used to 
tackle this classification problem.  Despite having a hierarchical structure, 
these generative models have a bag of words assumption. Due to this fact, 
they tend to misclassify texts having sentiment in the form of phrases. LDA 
and it's extensions don't work properly with phrases. To tackle this situation, 
we propose an unsupervised approach to sentiment analysis using the topical n-grams model
which has been shown to be effective with phrases. We train the topical n-grams 
model using two topics i.e., positive and negative, list of positive and negative 
words, and rules to detect positive and negative phrases. New documents are then 
classified using this trained model. The system gives better results than the 
existing Joint Sentiment Topic model. We also propose an approach to generate 
list of positive and negative words using LDA.

\end{abstract}

\section{Introduction}


\indent LDA (Latent Dirichlet Allocation) as shown by Blei et al.~\shortcite{blei2003latent} 
is a generative model used to discover topics in a document collection. It gives two types of 
distributions as output, document-topic and word-topic distributions. The document-topic
distribution gives the proportion of topics in each document and the word-topic 
distribution gives the probability of a word being in each topic. LDA works on the 
principle of co-occurrence. It assumes that words tending to appear together belong 
to the same topic. JST (Joint Sentiment Topic) as explained by Lin and He~\shortcite{lin2009joint} 
is a probabilistic generative model which extends LDA and discovers both sentiment 
and topic simultaneously in a document collection. JST has shown promising results 
on binary sentiment classification. 

There are many extensions of the basic LDA model which try to combine both sentiment 
and topic to solve the problem of sentiment analysis. All these models including LDA 
have one underlying assumption which makes them unsuitable for text classification
purposes. They assume that each word is generated separately and independent of other 
words. This is essentially the bag of words assumption. However, text being a sequence 
of words, the correct meaning of the text cannot be understood by merely capturing
co-occurrences. In addition to this, we also need to consider collocation of words.
A phrase is a collocation of words which usually has more meaning than the individual
words making up that phrase. There is a subtle difference between a phrase and collocation
of words. Not all collocations of words can be considered as a phrase. We need a 
model which takes into account phrases to completely understand the meaning of the text.

Topical n-grams model proposed by Wang et al.~\shortcite{wang2007topical} is one such generative 
model which takes into account not only co-occurrences but also collocations of words. It 
also decides whether a particular collocation of words should be considered as a phrase or not. 
We train this model using 2 topics viz. positive and negative, using a prior list of positive 
and negative words, and some rules to identify the subjective nature of phrases. The phrases
in our experiments are restricted to bigrams. We then use this trained model to infer the topic
distribution for new documents. The topic having higher proportion is considered to be
the class of the document.

Rest of the paper is organized as follows. Section~\ref{survey} discusses related work
in this area. Section~\ref{process} explains our approach in detail. The experimental 
setup is explained in Section~\ref{experiments}. Results of the experiments are presented
in Section~\ref{results}. Section~\ref{discussion} discusses these results followed by
explanation of the use of topic models for resource generation in Section~\ref{resource}.
Section~\ref{conclusion} concludes and  hints to some future work are given in 
Section~\ref{futurework}.


\section{Related Work}\label{survey}

There are many supervised, semi-supervised and unsupervised approaches to solve the 
sentiment analysis problem. One supervised method based on n-gram analysis is explained by
Bespalov et al.~\shortcite{bespalov2011sentimen}. In this they map the n-grams to a low-
dimensional latent semantic space where a classification function can be defined. Our approach 
being unsupervised, we will go through some of the related work in that direction. One more 
motivation to consider only semi-supervised and unsupervised approaches is the fact that they 
don't need any corpus and don't have any domain specific limitations. Rule based systems can 
also be considered as unsupervised systems but usually due to exceptions to these rules, their 
performance is affected. Due to this, we are not considering them in further discussion.

Turney and Littman~\shortcite{turney2002thumbs} used an unsupervised learning algorithm 
based on mutual information between document phrases and a small set of positive/negative 
paradigm words called seed words to classify the semantic orientation at the word/phrase 
level. Another unsupervised approach to classify the text at document level was proposed
by Turney~\shortcite{turney2002unsupervised}. Eguchi and Lavrenko~\shortcite{eguchi2006sentiment} 
created a generative model that jointly models sentiment words, topic words and sentiment
polarity in a sentence as a triple. Mei et al.~\shortcite{mei2007topic} proposed another 
generative model, called TSM (Topic Sentiment Mixture) model which can be used to discover 
topics in blogs as well as their associated sentiments. A novel generation model that 
unifies topic-relevance and opinion generation by a quadratic combination was proposed by 
Zhang and Ye~\shortcite{zhang2008generation}. A probabilistic generative model based on LDA
called JST (Joint Sentiment Topic) model was shown to perform well for sentiment analysis 
of reviews by Lin and He~\shortcite{lin2009joint}. It is a fully unsupervised method
and shows good result when priors are used for training. Another extension of LDA which tries
to unify aspect and sentiment was proposed by Jo and Oh~\shortcite{jo2011aspect}. All the 
unsupervised methods using generative models discussed here operate at the word level. Due 
to this they lose out on the information provided by phrases which may lead to incorrect 
classification.

\section{Phrase based Sentiment Analysis}\label{process}

We will explain the use of basic LDA and Topical n-grams for Sentiment Classification in
this section. We won't go into the mathematical details of these models for the constraint 
of space. Let us first list the basic steps to use any topic model for discovering topics.

\subsection{Using Topic models}

\begin{itemize}
 \itemsep0em
 \item Set number of topics.
 \item Remove stop-words as they do not belong to any topic.
 \item Estimate probabilities using some inference method.
 \item Use the trained model for inference of topics in new documents.
\end{itemize}

\subsection{Using LDA for Sentiment Classification}

To use basic LDA as a sentiment classifier, we add one more step to remove objective words.
Also, usually during Gibbs sampling the first step assigns topics randomly to words. Instead, we
introduce a prior information about the positivity and negativity of words to assign topics to 
words initially. The steps are as follows.

\begin{itemize}
 \itemsep0em
 \item Set number of topics, 2 in this case viz. positive and negative.
 \item Remove stop-words.
 \item Remove objective words as they won't affect sentiment.
 \item Gibbs Sampling with prior using lists of positive and negative words.
 \item Use the trained model to classify a new document as positive or negative.
\end{itemize}

\subsection{Using Topical n-grams Model for Sentiment Classification}

To make use of topical n-grams model for sentiment classification, we use a similar approach.

\begin{itemize}
 \itemsep0em
 \item Set number of topics equal to 2.
 \item Remove stop-words.
 \item Remove objective words as they won't affect sentiment. The objective words in this
 case do not include the negation words like \textit{don't, doesn't, won't, no}, etc. This is to
 ensure that we can catch negation of polarity when they are used with subjective words.
 \item Apply Gibbs Sampling with prior. The prior used in this case is more sophisticated and can
 handle both words and phrases. In case of words, it simply uses a list of positive and negative
 words. There are some rules to detect and assign topics to phrases which are explained next.
 \item Use the trained model to classify a new document as positive or negative.
\end{itemize}

\subsubsection*{Rules for Topic assignment of phrases}

At present, our rules are restricted to bigrams. We plan to extend them as explained in Section~\ref{futurework}.
In the following rules, we mean topic when we say polarity. The use of polarity makes it easy to understand
the rules as they are concerned with subjectivity.

\begin{enumerate}
 \itemsep0em
 \item If the first word in the bigram is a negation word and the second word is subjective then the polarity
 of the bigram is opposite to the polarity of the second word. \\
 \textbf{Examples:} \textit{won't like, won't regret, etc.}. Here, \textit{won't like} is assigned negative
 polarity and \textit{won't regret} is assigned positive polarity.
 \item If both the words in the bigram are subjective then are two cases. If both words are of the same polarity
 then resultant polarity is the same. But if their polarities are different, then the polarity of the first word
 is assigned to the bigram. \\
 \textbf{Examples:} \textit{beautifully amazing} is positive as both words as positive. \textit{lack respect} is
 assigned negative as per the rules.
\end{enumerate}

\section{Experimental Setup}\label{experiments}

Analysis was performed for the binary sentiment classification task. The language used in this case was English.
We conducted experiments on 4 models, BOW using SVM, LDA ~\cite{blei2003latent}, JST ~\cite{lin2009joint}, and 
Topical n-gram model ~\cite{wang2007topical}. We used two settings for the topic models, with and without prior. 
The word lists used are those specified in ~\cite{liu2010sentiment}. The implementations of SVM, LDA and Topical
n-gram in Mallet \footnote{http://mallet.cs.umass.edu/} have been used for evaluation. For JST, we have used the 
implementation provided by the authors \footnote{https://github.com/linron84/JST}. The default settings for the
hyper-parameters were used in all these implementations. Some changes in the implementations of LDA and Topical
n-gram were made to take into account the prior information.

\subsection*{Dataset}

To create the dataset we used the amazon reviews dataset provided by SNAP \footnote{http://snap.stanford.edu/}.
These reviews are not tagged with sentiment but they have ratings from 1 to 5. We used this information to create
a sentiment tagged corpus. The reviews with ratings less than 3 were tagged as negative and others were tagged
as positive. We conducted the experiments on 6,00,000 reviews containing equal number of positive and negative
reviews.

\section{Results}\label{results}

The results presented here are for 10-fold cross validation. For the topic models, due to the randomness involved 
during sampling, the best result obtained for each fold has been used for calculating the average. The Bag of words 
system performs better than all the models when no prior information is provided. The performance of topic models 
significantly increases when prior information is provided. Our approach to use Topical n-gram outperforms all the 
systems when a prior is used.

\begin{table}[h]
\begin{center}
\begin{tabular}{|l|c|}
\hline \bf System & \bf Avg. accuracy (\%)\\ \hline
BOW-SVM & 82.45\\
LDA & 65.34\\
LDA with prior & 80.19\\
JST & 68.64\\
JST with prior & 84.43\\
Topical n-gram & 63.57\\
Topical n-gram with prior & \textbf{87.32}\\
\hline
\end{tabular}
\end{center}
\caption{\label{result-table} Evaluation of systems}
\end{table}

\section{Discussion}\label{discussion}

The performance of our system is better due to the capacity to handle phrases. All the other systems, consider each word
separately. The performance improvement over JST is 3\% which is statistically significant. Though the system performs
better for this dataset, it still has some obvious limitations. The system highly depends on the rules used for
initial assignment of topics which is evident from the results. These rules don't apply to all the bigrams. Let us
consider, the bigram \textit{insanely good}. In this case, the first word is negative and second word is positive. The rules
will assign a negative polarity to this bigram. But in this phrase \textit{insanely} is used to increase the intensity of 
\textit{good}. The rules apply only for bigrams, we need to add more rules to handle n-grams. A phrase like \textit{I don't
think it's good} won't be handled by the system. During error analysis, we also found that some words like \textit{engrossing,
blockbuster, bravo, etc.}, are not present in the word lists. These words convey a positive sentiment but our system fails
to correctly classify reviews containing such words. Also, words like \textit{awsome} which is spelling mistake of \textit{awesome}
are often found in reviews. We thought that the accuracy might be increased if we could detect the correct subjective
nature of these words and also the bigrams using them. For this, we propose an approach for resource generation using LDA.

\section{Resource generation using LDA}\label{resource}

LDA can be used for resource generation of positive and negative words. The steps to do so are explained below.

\begin{itemize}
 \itemsep0em
 \item Set number of topics equal to 3 i.e., positive, negative and objective. We do not remove the objective
 words in this case as we want to find out which of them are positive or negative.
 \item Remove stop-words.
 \item Gibbs Sampling with prior using lists of positive and negative words. In the initial step, words present 
 in the list are assigned that specific topic but the other words as assigned a topic randomly.
 \item Get the top words in the positive and negative topics.
\end{itemize}

The word lists we generated using this approach were used to test the systems using priors. The results for the same
are shown in Table 2. We can see a marginal increase in the accuracy in this setting.

\begin{table}[h]
\begin{center}
\begin{tabular}{|l|c|}
\hline \bf System & \bf Avg. accuracy (\%)\\ \hline
LDA with prior & 80.21\\
JST with prior & 86.37\\
Topical n-gram with prior & \textbf{89.83}\\
\hline
\end{tabular}
\end{center}
\caption{\label{result-table} Evaluation of systems}
\end{table}

\section{Conclusion}\label{conclusion}

This paper made use of topical n-gram model for the binary sentiment classification problem. The motivation behind this
approach was to use not only words but also phrases to classify documents. The model was trained with documents containing
only subjective and negation words and bigrams formed by the combination of these. It also made use of rules to 
assign topics to words and bigrams in the intialization stage of estimation. Results using a prior show statistically 
significant improvement over the JST model. The paper also shows the use of LDA for resource generation, prompted by 
observations made during error analysis. The system shows marginal improvement in accuracy after using the resources
generated by this technique.

\section{Future Work}\label{futurework}

The system is focused on bigrams at the moment. It can be extended to handle n-grams by adding rules to detect and assign
topics. Instead of adding rules, we can make use of machine learning techniques detect the subjective nature of a phrase.

% include your own bib file like this:
%\bibliographystyle{acl}
%\bibliography{acl2014}

\begin{thebibliography}{}

\bibitem[\protect\citename{Bespalov et al.}2011]{bespalov2011sentimen}
Bespalov, Dmitriy and Bai, Bing and Qi, Yanjun and Shokoufandeh, Ali
\newblock 2011.
\newblock {\em Sentiment classification based on supervised latent n-gram analysis}.
\newblock {\em Proceedings of the 20th ACM international conference on Information and knowledge management}, pages 375-382.
\newblock ACM.

\bibitem[\protect\citename{Jo and Oh}2011]{jo2011aspect}
Jo, Yohan and Oh, Alice H
\newblock 2011.
\newblock {\em Aspect and sentiment unification model for online review analysis}.
\newblock {\em Proceedings of the fourth ACM international conference on Web search and data mining}, pages 815-824.
\newblock ACM.

\bibitem[\protect\citename{Zhang and Ye}2008]{zhang2008generation}
Zhang, Min and Ye, Xingyao
\newblock 2008.
\newblock {\em A generation model to unify topic relevance and lexicon-based sentiment for opinion retrieval}.
\newblock {\em Proceedings of the 31st annual international ACM SIGIR conference on Research and development in information retrieval}, pages 411-418.
\newblock ACM.

\bibitem[\protect\citename{Mei et al.}2007]{mei2007topic}
Mei, Qiaozhu and Ling, Xu and Wondra, Matthew and Su, Hang and Zhai, ChengXiang
\newblock 2007.
\newblock {\em Topic sentiment mixture: modeling facets and opinions in weblogs}.
\newblock {\em Proceedings of the 16th international conference on World Wide Web}, pages 171-180.
\newblock ACM.

\bibitem[\protect\citename{Eguchi and Lavrenko}2006]{eguchi2006sentiment}
Eguchi, Koji and Lavrenko, Victor
\newblock 2006.
\newblock {\em Sentiment retrieval using generative models}.
\newblock {\em Proceedings of the 2006 conference on empirical methods in natural language processing}, pages 345-354.
\newblock ACM.

\bibitem[\protect\citename{Peter D. Turney}2002]{turney2002thumbs}
Turney, Peter D
\newblock 2002.
\newblock {\em Thumbs up or thumbs down?: semantic orientation applied to unsupervised classification of reviews}.
\newblock {\em Proceedings of the 40th annual meeting on association for computational linguistics}, pages 417-424.
\newblock Association for Computational Linguistics.

\bibitem[\protect\citename{Turney and Littman}2002]{turney2002unsupervised}
Turney, Peter and Littman, Michael L
\newblock 2002.
\newblock {\em Unsupervised learning of semantic orientation from a hundred-billion-word corpus}.
\newblock {\em Handbook of natural language processing}, volume-2, pages 627-666.
\newblock CoRR, cs.LG/0212012.

\bibitem[\protect\citename{Bing Liu}2010]{liu2010sentiment}
Liu, Bing
\newblock 2010.
\newblock {\em Sentiment analysis and subjectivity}.
\newblock {\em Handbook of natural language processing}, volume-2, pages 627-666.
\newblock Chapman \& Hall.

\bibitem[\protect\citename{Wang et al.}2007]{wang2007topical}
Wang, Xuerui and McCallum, Andrew and Wei, Xing.
\newblock 2007.
\newblock {\em Topical n-grams: Phrase and topic discovery, with an application to information retrieval}.
\newblock {\em Data Mining, 2007. ICDM 2007. Seventh IEEE International Conference on}, pages 697-702.
\newblock IEEE.

\bibitem[\protect\citename{Lin and He}2009]{lin2009joint}
Lin, Chenghua and He, Yulan.
\newblock 2009.
\newblock {\em Joint sentiment-topic model for sentiment analysis}.
\newblock {\em Proceedings of the 18th ACM conference on Information and knowledge management}, pages 375-384.
\newblock ACM.

\bibitem[\protect\citename{Blei et al.}2003]{blei2003latent}
Blei, David M and Ng, Andrew Y and Jordan, Michael I.
\newblock 2003.
\newblock {\em Latent dirichlet allocation}
\newblock {\em The Journal of machine learning research}, volume-3, pages 993-1022.
\newblock JMLR.

\end{thebibliography}

\end{document}
